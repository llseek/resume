% LaTeX file for resume
% This file uses the resume document class (res.cls)

\documentclass{res}
%\usepackage{helvetica} % uses helvetica postscript font (download helvetica.sty)
%\usepackage{newcent}   % uses new century schoolbook postscript font
\usepackage{graphicx}
\setlength{\textheight}{10in} % increase text height to fit on 1-page


\begin{document}

\name{YUAN Xiaojie\\[12pt]}     % the \\[12pt] adds a blank
				        % line after name

\address{\bf  PRESENT ADDRESS\\Juli Road, Pudong New District,\\ Shanghai, China}
\address{\bf CONTACT INFORMATION \\Mail: yxj0207@gmail.com \\Tel: (+86)18915778082 \\Homepage: www.llseek.com}


\begin{resume}



\section{EDUCATION}
\vspace{0.03in}
\moveleft\hoffset\vbox{\hrule width\resumewidth height 1pt}
   \vspace{-0.1in}
   {\bf Shanghai Jiao Tong University} {\hfill 2009.9 - 2013.8\\}
   \emph{Bachelor of Electrical \& Computer Engineering\\}
   Major GPA: 3.25/4.0\\
   CET-6: 583/710


\section{COMPUTER SKILLS}
\vspace{0.03in}
\moveleft\hoffset\vbox{\hrule width\resumewidth height 1pt}
   \vspace{-0.1in}
    Proficient in C programming in Linux environment\\
    Proficient in development tools like Vim and Git\\
    Experienced with programming in Bash, Python and Makefile\\
    Familiar with embedded Linux development flow\\
    Familiar with Linux IIO/UVC/PM subsystems\\
    Familiar with commonly-used Design Patterns


\section{WORK EXPERIENCE}
\vspace{0.03in}
\moveleft\hoffset\vbox{\hrule width\resumewidth height 1pt}
   \vspace{-0.1in}
   {\bf Sony(China) Coporation Co., Ltd} {\hfill 2014.7 - present\\}
   \emph{Senior Linux Kernel/Driver Engineer\\}
   Responsible for design and development for sensor-related Linux driver and middleware

   {\bf Kingtrust Technologies Co., Ltd.} {\hfill 2013.8 - 2014.5\\}
   \emph{Firmware Engineer\\}
   Responsible for STM32-based firmware development for RFID card reader


\section{RESEARCH AND PROJECT}
\vspace{0.03in}
\moveleft\hoffset\vbox{\hrule width\resumewidth height 1pt}
   \vspace{-0.1in}
   {\bf Development of sensor middleware for Vesper platform,} {\hfill 2015.1 - 2015.5, 2015.11 - 2016.4\\}
   \emph {``Sensor Service"}\\
Implement plugin which interacts with IIO sensors and sensorhub library.\\
Implement plugin which replays GNSS/TRAM data from regular file.\\
Perform integration test with service layer and enhance RPC performance.\\
Implement unittest for Sensor Service with Python.\\
Implement sample clients for application team to use.
%~\\

   {\bf Development and performance tunning for BMI160 driver,} {\hfill 2014.9 - 2016.7\\}
   \emph {``BMI160 Driver"}\\
Implement BMI160 driver based on Linux IIO framework.\\
Add 'force read H/W fifo' feature to reduce latency of reading sensor data.\\
Implement suspend/resume functionality to reduce power consumption.
%~\\

   {\bf Feature implementation for UVC Gadget driver} {\hfill 2015.6 - 2015.10\\}
   \emph {``UVC Gadget Driver"}\\
Backport configfs feature from mainline kernel to linux-3.10.\\
Configure and enable super-speed bulk mode UVC gadget.\\
Fix 'last pixel missing' issue in each video frame.\\
Add YUV420 format support for uvcvideo and vivi driver.\\
Optimize program to forward video frames from on-board cameras to UVC gadget driver.
%~\\

   {\bf Development of firmware for RFID card reader} {\hfill 2014.2 - 2014.5\\}
   \emph {``RFID Reader Firmware"}\\
Implement driver for PN512 which performs contactless communication with RFID cards.\\
Implement peripheral drivers like TFT screen and fingerprint module.\\
Discuss with hardware engineer to optimize PCB layout.
%~\\

   {\bf Capstone Project Sponsored by Intel, Shanghai,} {\hfill 2012.9 - 2012.12\\}
   \emph {``Video Codec Auto Testing (Phase II)"}\\
Develop a pipelined video codec auto-testing tool with C++ in Linux environment.\\
Make use of OSS libraries including Xlib, OpenCV, Libdecodeqr to do image processing.\\
Introduce QR code technique to realize video synchronization and frame-by-frame comparing.
%~\\
%
%    {\bf Cross Compilaion of a Linux System,} {\hfill 2012.8\\}
%    \emph{``Linux From Scratch''}\\
%Cross compiled a tiny and bootable Linux system from downloaded tarballs including Binutils-2.22, Gcc-4.1.2, Glibc-2.5.1, Linux-2.6.25 kernel and etc.\\
%The host system is GNU/Linux with Intel x86 microprocessor.
%Carefully read through a 300-page manual and configure,compile,install step by step.
%~\\

%   {\bf Algorithm \& Data Structure Course Project,} \emph{Programmer} {\hfill Fall 2012\\}
%    \emph {``Design of a Simple Database"}\\
%Implemented a basic 2-3 tree which supports basic operations like insert, search, remove and list.\\
%Designed a simple database system storing city information: population, net worth and name.\\
%~\\

%   {\bf Computer Networks Course Project,} \emph{Team Leader} {\hfill Summer 2012\\}
%   \emph {``Simulation of MAC Protocols"}\\
%Built the basic model based on C++ for simulating protocols of pure ALOHA and Slotted ALOHA.\\
%Used Matlab to visualize the simulating result and did some relevant comparison and analysis work.\\
%~\\

%    {\bf Operating System Course Project,} {\hfill 2012.12\\}
%    \emph {``Simulation of OS Algorithms"}\\
%Made use of C++ to simulate the Round Robin Algorithm(responsible for process scheduling) and Banker's Algorithm(responsible for process resource distribution).
%
%    {\bf Computer Organization \& Design Course Project,} {\hfill 2011.10\\}
%    \emph {``Design of Pipelined Microprocessor"}\\
%Designed a five-stage pipelinded microprocessor with Verilog under Xilinx ISE 10.1.\\
%Implemented five stages%(IF, ID, EX, MEM and WB)
%, one forwarding unit and one hazard detection unit.\\
%Assembled all the components and debugged the general prototype.


\section{HONORS AND AWARDS}
\vspace{0.03in}
\moveleft\hoffset\vbox{\hrule width\resumewidth height 1pt}
   \vspace{-0.1in}
Excellent Graduate, SJTU {(top 10\%)} {\hfill 2013.6\\}
Third-Class Scholarship for Excellent Students, SJTU {(top 30\%~)}                     {\hfill 2012.6\\}
Silver Prize of Capstone Projects {(2/15 groups)}                                      {\hfill 2012.12}


\section{REFERENCE}
\vspace{0.03in}
\moveleft\hoffset\vbox{\hrule width\resumewidth height 1pt}
   \vspace{-0.1in}
Github: github.com/llseek

\end{resume}
\end{document}
