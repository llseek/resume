\documentclass[11pt,a4paper,noblankpage]{moderncv}

% moderncv themes
\moderncvtheme[blue]{classic}

% character encoding
\usepackage{color}
\definecolor{Blue}{rgb}{0.2,0.4,0.65}
\usepackage{fontspec,xunicode}
\setmainfont{WenQuanYi Zen Hei}
\usepackage[slantfont,boldfont]{xeCJK}
\usepackage{xcolor}
\setCJKmainfont{WenQuanYi Zen Hei}

% adjust the page margin
\usepackage[scale=0.85]{geometry}
\AtBeginDocument{\recomputelengths}

% personal data
\firstname{袁晓杰}
\familyname{}
%\address{}{上海市闵行区东川路800号}
\mobile{18915778082}
\email{yuanxj@yuanxj.com}
\nopagenumbers{}

% content
\begin{document}
\maketitle

\section{基本信息}
\cvcomputer {毕业院校:}{上海交通大学}{最高学历:}{本科}
\cvcomputer {专业方向:}{电子与计算机工程}{毕业时间:}{2013年8月}
\cvcomputer {户籍地址:}{江苏省苏州市}{个人博客:}{www.yuanxj.com}

\section{IT技能}
\cvline{\Neutral~}{4年Linux使用经验}
\cvline{\Neutral~}{熟练掌握C语言程序开发}
\cvline{\Neutral~}{了解C++/Shell/Perl/Makefile}
\cvline{\Neutral~}{良好的代码阅读能力和编码风格}
\cvline{\Neutral~}{熟悉嵌入式Linux开发环境和开发流程}
\cvline{\Neutral~}{熟悉UART, SPI, I2C, USB等总线协议}
\cvline{\Neutral~}{熟悉ISO-14443A/B等RFID/NFC通讯协议}
\cvline{\Neutral~}{具备基于Cortex-M3和Cortex-A8的驱动开发经验}
%\cvline{\Neutral~}{具备基于Cortex-M3的TFT屏/OLED屏/热敏打印机/指纹模块等外设的驱动开发经验}
%\cvline{\Neutral~}{具备基于Cortex-A8的GPIO/I2C/RTC驱动开发经验}

\section{工作经验}
\cvline{\Neutral~}{\textcolor{Blue}{固件开发工程师, 上海人行道网络信息技术有限公司} \hfill 2013.8 - 2014.5}
\cvline{\Neutral~}{负责通用微控制器及专用安全芯片的固件开发}
\cvline{\Neutral~}{驱动/应用程序的代码实现,独立完成板级调试,协助团队完成系统调试}
\cvline{\Neutral~}{编写项目进程中的相关文档,包括相关项目文档和测试文档等}

\section{项目经验}
\cvline{\Neutral~}{\textcolor{Blue}{基于Cortex-M3的多功能RFID USB读卡器} \hfill 2014年2月 - 2014年5月}
\cvline{项目介绍}{此款RFID读卡器使用NXP的PN512读卡芯片,支持ISO-14443A/B协议,支持固件IAP升级;读卡器与上位机程序之间以自定义格式的协议报文通过USB接口进行通讯;读卡器固件采用RT-Thread(一种开源RTOS)进行线程调度与线程同步}
\cvline{项目职责}{负责PN512读卡芯片驱动,USB CDC虚拟串口驱动和固件IAP模块的板级开发调试}

\cvline{\Neutral~}{\textcolor{Blue}{基于Cortex-A8的嵌入式Linux驱动开发} \hfill 2013年12月 - 2014年5月}
\cvline{项目介绍}{此项目为个人业余项目,主要目的是学习和巩固嵌入式Linux系统外设的驱动开发}
\cvline{项目职责}{搭建嵌入式开发环境,包括安装交叉工具链,配置串口/TFTP/NFS/ssh等;为Beaglebone目标板交叉编译U-Boot和Linux-3.12.10内核;开发基于内核GPIO子系统用于控制外部蜂鸣器的内核模块;结合内核的设备模型分析DS1307驱动源码和框架}

\cvline{\Neutral~}{\textcolor{Blue}{基于Cortex-M3的指纹模块测试与评估} \hfill 2013年12月 - 2014年1月}
\cvline{项目介绍}{此款指纹模块为公司某款安全产品准备采用的信息采集模块;此项目主要针对此指纹模块进行测试与评估,以决定方案的可行性}
\cvline{项目职责}{依据Datasheet和Spec文档开发SPI接口的指纹模块驱动;驱动程序实现了检测手指接触的指纹自动采集功能和指纹数据上传功能;另外在Linux上位机端开发了基于串口的指纹接收与基于OpenCV的指纹图像清晰化处理程序}

\cvline{\Neutral~}{\textcolor{Blue}{基于Cortex-M3的TFT液晶屏驱动和人机界面开发} \hfill 2013年9月 - 2013年10月}
\cvline{项目介绍}{此TFT液晶屏主要应用于编码采集器上,显示采集器中存储的菜名和当天单价等信息,用户可以使用采集器上的专用键盘切换界面}
\cvline{项目职责}{编写TFT屏底层驱动;根据美工提供的界面图片实现人机界面的显示与切换;使用TFT液晶屏控制器提供的缓冲区加快屏幕刷新速度以弥补硬件设计中的不足}

\cvline{\Neutral~}{\textcolor{Blue}{基于Linux的视频Codec自动测试工具开发} \hfill 2012年9月 - 2012年12月}
\cvline{项目介绍}{此项目为Intel SSG赞助的毕业设计项目,目的是开发一款视频Codec的自动测试工具;使用OpenCV/Decodeqr/Jpeg等开源库对X11捕获的视频帧进行提取特征值等一系列处理;依据从二维码解析出来的信息对视频帧进行同步与比较,以确定Codec优劣与否;项目中使用的版本控制工具为git}
\cvline{项目职责}{负责OpenCV图像处理模块和pthread多线程框架的代码实现}

%\cvline{\Neutral~}{\textcolor{Blue}{编译小型Linux系统(LFS)} \hfill 2012年8月}
%\cvline{介绍}{在Fedora 17环境下使用Linux-Kernel-3.5.2、Binutils-2.22、Gcc-4.7.1、Glibc-2.16.0等源码包和一些必要的补丁编译个一个小型的可启动Linux系统,boot loader为GRUB-2.00}
%\cvline{职责}{通读300多页的LFS手册,逐步配置-编译-安装}

%\cvline{\Neutral~}{\textcolor{Blue}{操作系统算法模拟} \hfill 2012年12月}
%\cvline{介绍}{使用C++模拟了分时系统中常见的负责进程调度的时间片轮转算法(进程轮流占有CPU)和负责进程资源分配的银行家算法(防止进程死锁)}
%\cvline{职责}{使用循环链表实现进程轮转,使用二维数组模拟资源状态}

%\cvline{\Neutral~}{\textcolor{Blue}{流水线处理器设计} \hfill 2011年10月}
%\cvline{介绍}{在Xilinx ISE平台下使用Verilog硬件描述语言实现了一个支持MIPS指令集子集的五阶段(IF-ID-EX-MEM-WB)流水线处理器}
%\cvline{职责}{负责IF/ID/EX三个阶段和一个Hazard检测单元的代码实现}

%\section{实习经历}
%\cvline{\Neutral~}{\textcolor{Blue}{实习生, 上海泛腾电子科技有限公司} \hfill 2013年3月 - 2013年5月}
%\cvline{职责}{仿照ClearOS(一种适用于中小企业的网络和网关服务器系统),针对Tilera众核平台交叉编译30多个应用程序源码包(包括防病毒,防垃圾邮件,VPN等软件);解决编译过程中由于源码底层缺少对Tilera处理器的支持出现的各种编译错误;将编译完的每个软件制作成RPM包,交由RPM包管理工具自动管理}

\section{奖励情况}
\cvline{\Neutral~}{上海交通大学优秀毕业生(校级)  \hfill 2013年6月}
\cvline{\Neutral~}{上海交通大学三等奖学金(前30\%)\hfill 2012年9月}
\cvline{\Neutral~}{毕业设计项目组银奖(共15个项目组) \hfill 2012年12月}

\end{document}
